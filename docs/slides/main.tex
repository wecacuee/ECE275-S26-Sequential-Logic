\documentclass[oneside]{book}
\usepackage[utf8]{inputenc}
\usepackage[top=1in]{geometry}
\usepackage{graphicx}
\usepackage{booktabs}
\usepackage{amsmath}
\usepackage{amsthm}
\usepackage[only]{excludeonly}
\usepackage{tikz}
\usepackage{verbatim}
\usetikzlibrary{circuits.logic.US,positioning,calc} 
\usetikzlibrary{matrix,shapes,arrows,fit,tikzmark}
\usepackage[american]{circuitikz}
\usepackage{enumitem,amssymb}
\usepackage{url}
\usepackage{wrapfig}
\usepackage[disable]{easy-todo}
\usepackage{multirow}
%\usepackage{todonotes}

\newlist{todolist}{itemize}{2}
\setlist[todolist]{label=$\square$}

\newcommand{\hilight}[1]{\setlength{\fboxsep}{1pt}\colorbox{red}{#1}}

\usepackage{pifont}
\newcommand{\cmark}{\ding{51}}%
\newcommand{\xmark}{\ding{55}}%
\newcommand{\done}{\rlap{$\square$}{\raisebox{2pt}{\large\hspace{1pt}\cmark}}%
  \hspace{-2.5pt}}
%\newcommand{\important}{\hilight{\rlap{$\square$}{\raisebox{2pt}{\large\hspace{1pt}\cmark}}}%
%  \hspace{-2.5pt}}
\newcommand{\wontfix}{\rlap{$\square$}{\large\hspace{1pt}\xmark}}

\input{0928-quine-mccluskey/karnaugh}
\input{0907-boolean-algebra/sym}
\graphicspath{{0831-number-system/}
{0831-number-system}
{0907-boolean-algebra}
{0916-K-maps}
{homeworks/hw1}
{homeworks/hw2}
{labs/01-quartus-setup/}
 {labs/02-verilog-mux/}
{0928-adders-carry-ahead}
{0928-quine-mccluskey}
{0930-review}
{1014-analog-details}
{1024-cmos-gate-review}
{1026-sequential-logic}
{1031-review}
{1121-mealy-moore-seq-detector}
{11-29-notes.pdf}
{1130-FSM-optimization}
{1205-mux-decoder}
{1207-more-definitions}
} 
\title{Digital circuit design notes}
\author{Vikas Dhiman for ECE275\footnote{The notes are from the following books \cite{harris2022digital,stephen2022fundamentals}. Not intended for distribution outside the class.}}
\newtheorem{example}{Example}[chapter]
\newtheorem{prob}{Problem}[chapter]
\newtheorem{remark}{Remark}[chapter]
\newtheorem{definition}{Definition}[chapter]

\newcommand{\bw}{\bar{w}}
\newcommand{\bx}{\bar{x}}
\newcommand{\by}{\bar{y}}
\newcommand{\bz}{\bar{z}}
\newcommand{\bW}{\bar{W}}
\newcommand{\bX}{\bar{X}}
\newcommand{\bY}{\bar{Y}}
\newcommand{\bZ}{\bar{Z}}
\newcommand{\bA}{\bar{A}}
\newcommand{\bB}{\bar{B}}
\newcommand{\bC}{\bar{C}}
\newcommand{\bD}{\bar{D}}

\newcommand{\cred}{\color{red}}
\newcommand{\cg}{\color{green!60!black}}
\newcommand{\cb}{\color{blue}}
% Some options common to all the nodes and paths
\tikzset{   
  every picture/.style={remember picture,baseline},
  every node/.style={anchor=base,align=center,outer sep=1.5pt},
  every path/.style={thick},
}

\newcommand\marktopleft[1]{%
  \tikz[overlay,remember picture] 
  \node (marker-#1-a) at (.3em,.3em) {};%
}
\newcommand\markbottomright[2]{%
  \tikz[overlay,remember picture] 
  \node (marker-#1-b) at (.1em,.3em) {};%
  \tikz[overlay,remember picture,inner sep=1pt]
  \node[draw={#2},rounded corners,fit=(marker-#1-a.north west) (marker-#1-b.south east)] {};%
}
\newcommand\markpolytopleft[1]{%
  \tikz[overlay,remember picture]%
  \node (marker-#1-tl) at (.3em,.3em) {};%
}
\newcommand\markpolytopright[1]{%
  \tikz[overlay,remember picture]%
  \node (marker-#1-tr) at (.3em,.3em) {};%
}
\newcommand\markpolybottomright[1]{%
  \tikz[overlay,remember picture] 
  \node (marker-#1-br) at (.3em,.3em) {};%
  }
\newcommand\markpolybottomleft[2]{%
  \tikz[overlay,remember picture]
  \node (marker-#1-bl) at (.3em,.3em) {};%
  \tikz[overlay,remember picture,inner sep=1pt]
  \path [draw={#2},rounded corners]
  (marker-#1-tl.north west) --
  (marker-#1-tl.north east) --
  (marker-#1-bl.south east) --
  (marker-#1-br.south west);%
}

\newcommand{\notescol}{black}

\input{labs/01-quartus-setup/lab-preamble.tex}

\begin{document}
\maketitle

\tableofcontents
\newpage
\listoftodos

%\todoi{Weave homeworks, labs and notes}
%\todoi{Less emphasis on K-maps, more on Verilog}
%\todoi{Maybe quine mccluskey and espresso by C-coding example}

\newpage


\chapter{Boolean Algebra}
\input{./0907-boolean-algebra/0907-notes.tex}
\input{./0907-boolean-algebra/0912-notes.tex}
% \chapter{Sets, functions and relations}
% \input{./0906-sets-functions-relations/notes.tex}
%\input{./0907-boolean-algebra/0912-fpga-notes.tex}
% \chapter{Verilog and Schematics}
% \input{labs/01-quartus-setup/body.tex}
% \input{labs/02-verilog-mux/body.tex}
\chapter{Logic minimization}
\input{./0916-K-maps/0919-notes.tex}
\input{./0916-K-maps/0921-notes.tex}
\input{./0916-K-maps/0923-notes.tex}
\input{./homeworks/hw2/hw2-contents.tex}
\chapter*{Review}
\input{./0930-review/0928-syllabus-covered.tex}
\chapter{Sample midterm exam}
\input{./0930-review/0930-sample-exam.tex}
\chapter{Number System}
\input{./0831-number-system/0831-notes.tex}
\input{./0831-number-system/0902-notes.tex}
\input{./homeworks/hw1/hw1-contents.tex}
\input{./0928-quine-mccluskey/0928-nand-nor-gates.tex}
\chapter{Muxes and Decoders}
\input{1205-mux-decoder/document.tex}
\chapter{Sequential Logic}
\input{1026-sequential-logic/document.tex}
%\chapter{Review 2}
%\input{./1031-review/1031-syllabus-covered.tex}
%\input{./1031-review/20221109-midterm.tex}
%\chapter{Mealy Moore Sequence detector}
%\input{1121-mealy-moore-seq-detector/document.tex}
\chapter{Finite State Machine Optimization}
\input{1130-FSM-optimization/document}
\chapter*{Review}
\input{./1026-sequential-logic/1208-syllabus-covered.tex}
\chapter{Sample Midterm 2}
\input{20250407-midterm2/20231215-final.tex}
% %\input{./1205-mux-decoder/2023-12-01-syllabus-covered.tex}
\chapter{Analog details}
\input{1014-analog-details/document.tex}
\chapter{More definitions}
\input{1207-more-definitions/document.tex}
\chapter{Quine McCluskey}
\input{./0928-quine-mccluskey/0928-quine-mccluskey.tex}

\bibliography{1130-FSM-optimization/main.bib}
\bibliographystyle{plain}
\end{document}
